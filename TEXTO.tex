\begin{pages}

  \begin{Leftside}
    \beginnumbering
    \pstart

משֶׁה קִבֵּל תּוֹרָה מִסִּינַי, וּמְסָרָהּ לִיהוֹשֻׁעַ, וִיהוֹשֻׁעַ לִזְקֵנִים, וּזְקֵנִים לִנְבִיאִים, וּנְבִיאִים מְסָרוּהָ לְאַנְשֵׁי כְנֶסֶת הַגְּדוֹלָה. הֵם אָמְרוּ שְׁלשָׁה דְבָרִים, הֱווּ מְתוּנִים בַּדִּין, וְהַעֲמִידוּ תַלְמִידִים הַרְבֵּה, וַעֲשׂוּ סְיָג לַתּוֹרָה:

\pend
    \endnumbering
  \end{Leftside}

  \begin{Rightside}
    \beginnumbering
    \pstart
\lipsum[1]
\pend
    \endnumbering

  \end{Rightside}

\end{pages}
  \Pages


\chapter{Capítulo 1}

משֶׁה קִבֵּל תּוֹרָה מִסִּינַי, וּמְסָרָהּ לִיהוֹשֻׁעַ, וִיהוֹשֻׁעַ לִזְקֵנִים, וּזְקֵנִים לִנְבִיאִים, וּנְבִיאִים מְסָרוּהָ לְאַנְשֵׁי כְנֶסֶת הַגְּדוֹלָה. הֵם אָמְרוּ שְׁלשָׁה דְבָרִים, הֱווּ מְתוּנִים בַּדִּין, וְהַעֲמִידוּ תַלְמִידִים הַרְבֵּה, וַעֲשׂוּ סְיָג לַתּוֹרָה:

שִׁמְעוֹן הַצַּדִּיק הָיָה מִשְּׁיָרֵי כְנֶסֶת הַגְּדוֹלָה. הוּא הָיָה אוֹמֵר, עַל שְׁלשָׁה דְבָרִים הָעוֹלָם עוֹמֵד, עַל הַתּוֹרָה וְעַל הָעֲבוֹדָה וְעַל גְּמִילוּת חֲסָדִים:

אַנְטִיגְנוֹס אִישׁ סוֹכוֹ קִבֵּל מִשִּׁמְעוֹן הַצַּדִּיק. הוּא הָיָה אוֹמֵר, אַל תִּהְיוּ כַעֲבָדִים הַמְשַׁמְּשִׁין אֶת הָרַב עַל מְנָת לְקַבֵּל פְּרָס, אֶלָּא הֱווּ כַעֲבָדִים הַמְשַׁמְּשִׁין אֶת הָרַב שֶׁלֹּא עַל מְנָת לְקַבֵּל פְּרָס, וִיהִי מוֹרָא שָׁמַיִם עֲלֵיכֶם:
יוֹסֵי בֶן יוֹעֶזֶר אִישׁ צְרֵדָה וְיוֹסֵי בֶן יוֹחָנָן אִישׁ יְרוּשָׁלַיִם קִבְּלוּ מֵהֶם. יוֹסֵי בֶן יוֹעֶזֶר אִישׁ צְרֵדָה אוֹמֵר, יְהִי בֵיתְךָ בֵית וַעַד לַחֲכָמִים, וֶהֱוֵי מִתְאַבֵּק בַּעֲפַר רַגְלֵיהֶם, וֶהֱוֵי שׁוֹתֶה בְצָמָא אֶת דִּבְרֵיהֶם:
יוֹסֵי בֶן יוֹחָנָן אִישׁ יְרוּשָׁלַיִם אוֹמֵר, יְהִי בֵיתְךָ פָתוּחַ לִרְוָחָה, וְיִהְיוּ עֲנִיִּים בְּנֵי בֵיתֶךָ, וְאַל תַּרְבֶּה שִׂיחָה עִם הָאִשָּׁה. בְּאִשְׁתּוֹ אָמְרוּ, קַל וָחֹמֶר בְּאֵשֶׁת חֲבֵרוֹ. מִכָּאן אָמְרוּ חֲכָמִים, כָּל זְמַן שֶׁאָדָם מַרְבֶּה שִׂיחָה עִם הָאִשָּׁה, גּוֹרֵם רָעָה לְעַצְמוֹ, וּבוֹטֵל מִדִּבְרֵי תוֹרָה, וְסוֹפוֹ יוֹרֵשׁ גֵּיהִנֹּם:
יְהוֹשֻׁעַ בֶּן פְּרַחְיָה וְנִתַּאי הָאַרְבֵּלִי קִבְּלוּ מֵהֶם. יְהוֹשֻׁעַ בֶּן פְּרַחְיָה אוֹמֵר, עֲשֵׂה לְךָ רַב, וּקְנֵה לְךָ חָבֵר, וֶהֱוֵי דָן אֶת כָּל הָאָדָם לְכַף זְכוּת:
נִתַּאי הָאַרְבֵּלִי אוֹמֵר, הַרְחֵק מִשָּׁכֵן רָע, וְאַל תִּתְחַבֵּר לָרָשָׁע, וְאַל תִּתְיָאֵשׁ מִן הַפֻּרְעָנוּת:
יְהוּדָה בֶן טַבַּאי וְשִׁמְעוֹן בֶּן שָׁטָח קִבְּלוּ מֵהֶם. יְהוּדָה בֶן טַבַּאי אוֹמֵר, אַל תַּעַשׂ עַצְמְךָ כְעוֹרְכֵי הַדַּיָּנִין. וּכְשֶׁיִּהְיוּ בַעֲלֵי דִינִין עוֹמְדִים לְפָנֶיךָ, יִהְיוּ בְעֵינֶיךָ כִרְשָׁעִים. וּכְשֶׁנִּפְטָרִים מִלְּפָנֶיךָ, יִהְיוּ בְעֵינֶיךָ כְזַכָּאִין, כְּשֶׁקִּבְּלוּ עֲלֵיהֶם אֶת הַדִּין:
שִׁמְעוֹן בֶּן שָׁטָח אוֹמֵר, הֱוֵי מַרְבֶּה לַחְקֹר אֶת הָעֵדִים, וֶהֱוֵי זָהִיר בִּדְבָרֶיךָ, שֶׁמָּא מִתּוֹכָם יִלְמְדוּ לְשַׁקֵּר:
שְׁמַעְיָה וְאַבְטַלְיוֹן קִבְּלוּ מֵהֶם. שְׁמַעְיָה אוֹמֵר, אֱהֹב אֶת הַמְּלָאכָה, וּשְׂנָא אֶת הָרַבָּנוּת, וְאַל תִּתְוַדַּע לָרָשׁוּת:
אַבְטַלְיוֹן אוֹמֵר, חֲכָמִים, הִזָּהֲרוּ בְדִבְרֵיכֶם, שֶׁמָּא תָחוּבוּ חוֹבַת גָּלוּת וְתִגְלוּ לִמְקוֹם מַיִם הָרָעִים, וְיִשְׁתּוּ הַתַּלְמִידִים הַבָּאִים אַחֲרֵיכֶם וְיָמוּתוּ, וְנִמְצָא שֵׁם שָׁמַיִם מִתְחַלֵּל:
הִלֵּל וְשַׁמַּאי קִבְּלוּ מֵהֶם. הִלֵּל אוֹמֵר, הֱוֵי מִתַּלְמִידָיו שֶׁל אַהֲרֹן, אוֹהֵב שָׁלוֹם וְרוֹדֵף שָׁלוֹם, אוֹהֵב אֶת הַבְּרִיּוֹת וּמְקָרְבָן לַתּוֹרָה:
הוּא הָיָה אוֹמֵר, נָגֵד שְׁמָא, אָבֵד שְׁמֵהּ. וּדְלֹא מוֹסִיף, יָסֵף. וּדְלֹא יָלֵיף, קְטָלָא חַיָּב. וּדְאִשְׁתַּמֵּשׁ בְּתָגָא, חָלֵף:
הוּא הָיָה אוֹמֵר, אִם אֵין אֲנִי לִי, מִי לִי. וּכְשֶׁאֲנִי לְעַצְמִי, מָה אֲנִי. וְאִם לֹא עַכְשָׁיו, אֵימָתָי:
שַׁמַּאי אוֹמֵר, עֲשֵׂה תוֹרָתְךָ קֶבַע. אֱמֹר מְעַט וַעֲשֵׂה הַרְבֵּה, וֶהֱוֵי מְקַבֵּל אֶת כָּל הָאָדָם בְּסֵבֶר פָּנִים יָפוֹת:
רַבָּן גַּמְלִיאֵל הָיָה אוֹמֵר, עֲשֵׂה לְךָ רַב, וְהִסְתַּלֵּק מִן הַסָּפֵק, וְאַל תַּרְבֶּה לְעַשֵּׂר אֹמָדוֹת:
שִׁמְעוֹן בְּנוֹ אוֹמֵר, כָּל יָמַי גָּדַלְתִּי בֵין הַחֲכָמִים, וְלֹא מָצָאתִי לַגּוּף טוֹב אֶלָּא שְׁתִיקָה. וְלֹא הַמִּדְרָשׁ הוּא הָעִקָּר, אֶלָּא הַמַּעֲשֶׂה. וְכָל הַמַּרְבֶּה דְבָרִים, מֵבִיא חֵטְא:
רַבָּן שִׁמְעוֹן בֶּן גַּמְלִיאֵל אוֹמֵר, עַל שְׁלשָׁה דְבָרִים הָעוֹלָם עוֹמֵד, עַל הַדִּין וְעַל הָאֱמֶת וְעַל הַשָּׁלוֹם, שֶׁנֶּאֱמַר (זכריה ח) אֱמֶת וּמִשְׁפַּט שָׁלוֹם שִׁפְטוּ בְּשַׁעֲרֵיכֶם:

Chapter 2

רַבִּי אוֹמֵר, אֵיזוֹהִי דֶרֶךְ יְשָׁרָה שֶׁיָּבֹר לוֹ הָאָדָם, כֹּל שֶׁהִיא תִפְאֶרֶת לְעוֹשֶׂיהָ וְתִפְאֶרֶת לוֹ מִן הָאָדָם. וֶהֱוֵי זָהִיר בְּמִצְוָה קַלָּה כְבַחֲמוּרָה, שֶׁאֵין אַתָּה יוֹדֵעַ מַתַּן שְׂכָרָן שֶׁל מִצְוֹת. וֶהֱוֵי מְחַשֵּׁב הֶפְסֵד מִצְוָה כְּנֶגֶד שְׂכָרָהּ, וּשְׂכַר עֲבֵרָה כְנֶגֶד הֶפְסֵדָהּ. וְהִסְתַּכֵּל בִּשְׁלשָׁה דְבָרִים וְאִי אַתָּה בָא לִידֵי עֲבֵרָה, דַּע מַה לְּמַעְלָה מִמְּךָ, עַיִן רוֹאָה וְאֹזֶן שׁוֹמַעַת, וְכָל מַעֲשֶׂיךָ בַסֵּפֶר נִכְתָּבִין:
רַבָּן גַּמְלִיאֵל בְּנוֹ שֶׁל רַבִּי יְהוּדָה הַנָּשִׂיא אוֹמֵר, יָפֶה תַלְמוּד תּוֹרָה עִם דֶּרֶךְ אֶרֶץ, שֶׁיְּגִיעַת שְׁנֵיהֶם מְשַׁכַּחַת עָוֹן. וְכָל תּוֹרָה שֶׁאֵין עִמָּהּ מְלָאכָה, סוֹפָהּ בְּטֵלָה וְגוֹרֶרֶת עָוֹן. וְכָל הָעֲמֵלִים עִם הַצִּבּוּר, יִהְיוּ עֲמֵלִים עִמָּהֶם לְשֵׁם שָׁמַיִם, שֶׁזְּכוּת אֲבוֹתָם מְסַיַּעְתָּן וְצִדְקָתָם עוֹמֶדֶת לָעַד. וְאַתֶּם, מַעֲלֶה אֲנִי עֲלֵיכֶם שָׂכָר הַרְבֵּה כְּאִלּוּ עֲשִׂיתֶם:
הֱווּ זְהִירִין בָּרָשׁוּת, שֶׁאֵין מְקָרְבִין לוֹ לָאָדָם אֶלָּא לְצֹרֶךְ עַצְמָן. נִרְאִין כְּאוֹהֲבִין בִּשְׁעַת הֲנָאָתָן, וְאֵין עוֹמְדִין לוֹ לָאָדָם בִּשְׁעַת דָּחְקוֹ:
הוּא הָיָה אוֹמֵר, עֲשֵׂה רְצוֹנוֹ כִרְצוֹנְךָ, כְּדֵי שֶׁיַּעֲשֶׂה רְצוֹנְךָ כִרְצוֹנוֹ. בַּטֵּל רְצוֹנְךָ מִפְּנֵי רְצוֹנוֹ, כְּדֵי שֶׁיְּבַטֵּל רְצוֹן אֲחֵרִים מִפְּנֵי רְצוֹנֶךָ. הִלֵּל אוֹמֵר, אַל תִּפְרֹשׁ מִן הַצִּבּוּר, וְאַל תַּאֲמִין בְּעַצְמְךָ עַד יוֹם מוֹתְךָ, וְאַל תָּדִין אֶת חֲבֵרְךָ עַד שֶׁתַּגִּיעַ לִמְקוֹמוֹ, וְאַל תֹּאמַר דָּבָר שֶׁאִי אֶפְשָׁר לִשְׁמֹעַ, שֶׁסּוֹפוֹ לְהִשָּׁמַע. וְאַל תֹּאמַר לִכְשֶׁאִפָּנֶה אֶשְׁנֶה, שֶׁמָּא לֹא תִפָּנֶה:
הוּא הָיָה אוֹמֵר, אֵין בּוּר יְרֵא חֵטְא, וְלֹא עַם הָאָרֶץ חָסִיד, וְלֹא הַבַּיְשָׁן לָמֵד, וְלֹא הַקַּפְּדָן מְלַמֵּד, וְלֹא כָל הַמַּרְבֶּה בִסְחוֹרָה מַחְכִּים. וּבְמָקוֹם שֶׁאֵין אֲנָשִׁים, הִשְׁתַּדֵּל לִהְיוֹת אִישׁ:
אַף הוּא רָאָה גֻלְגֹּלֶת אַחַת שֶׁצָּפָה עַל פְּנֵי הַמַּיִם. אָמַר לָהּ, עַל דַּאֲטֵפְתְּ, אַטְפוּךְ. וְסוֹף מְטִיפַיִךְ יְטוּפוּן:
הוּא הָיָה אוֹמֵר, מַרְבֶּה בָשָׂר, מַרְבֶּה רִמָּה. מַרְבֶּה נְכָסִים, מַרְבֶּה דְאָגָה. מַרְבֶּה נָשִׁים, מַרְבֶּה כְשָׁפִים. מַרְבֶּה שְׁפָחוֹת, מַרְבֶּה זִמָּה. מַרְבֶּה עֲבָדִים, מַרְבֶּה גָזֵל. מַרְבֶּה תוֹרָה, מַרְבֶּה חַיִּים. מַרְבֶּה יְשִׁיבָה, מַרְבֶּה חָכְמָה. מַרְבֶּה עֵצָה, מַרְבֶּה תְבוּנָה. מַרְבֶּה צְדָקָה, מַרְבֶּה שָׁלוֹם. קָנָה שֵׁם טוֹב, קָנָה לְעַצְמוֹ. קָנָה לוֹ דִבְרֵי תוֹרָה, קָנָה לוֹ חַיֵּי הָעוֹלָם הַבָּא:
רַבָּן יוֹחָנָן בֶּן זַכַּאי קִבֵּל מֵהִלֵּל וּמִשַּׁמָּאי. הוּא הָיָה אוֹמֵר, אִם לָמַדְתָּ תוֹרָה הַרְבֵּה, אַל תַּחֲזִיק טוֹבָה לְעַצְמְךָ, כִּי לְכָךְ נוֹצָרְתָּ. חֲמִשָּׁה תַלְמִידִים הָיוּ לוֹ לְרַבָּן יוֹחָנָן בֶּן זַכַּאי, וְאֵלּוּ הֵן, רַבִּי אֱלִיעֶזֶר בֶּן הוֹרְקְנוֹס, וְרַבִּי יְהוֹשֻׁעַ בֶּן חֲנַנְיָה, וְרַבִּי יוֹסֵי הַכֹּהֵן, וְרַבִּי שִׁמְעוֹן בֶּן נְתַנְאֵל, וְרַבִּי אֶלְעָזָר בֶּן עֲרָךְ. הוּא הָיָה מוֹנֶה שִׁבְחָן. רַבִּי אֱלִיעֶזֶר בֶּן הוֹרְקְנוֹס, בּוֹר סוּד שֶׁאֵינוֹ מְאַבֵּד טִפָּה. רַבִּי יְהוֹשֻׁעַ בֶּן חֲנַנְיָה, אַשְׁרֵי יוֹלַדְתּוֹ. רַבִּי יוֹסֵי הַכֹּהֵן, חָסִיד. רַבִּי שִׁמְעוֹן בֶּן נְתַנְאֵל, יְרֵא חֵטְא. וְרַבִּי אֶלְעָזָר בֶּן עֲרָךְ, מַעְיָן הַמִּתְגַּבֵּר. הוּא הָיָה אוֹמֵר, אִם יִהְיוּ כָל חַכְמֵי יִשְׂרָאֵל בְּכַף מֹאזְנַיִם, וֶאֱלִיעֶזֶר בֶּן הוֹרְקְנוֹס בְּכַף שְׁנִיָּה, מַכְרִיעַ אֶת כֻּלָּם. אַבָּא שָׁאוּל אוֹמֵר מִשְּׁמוֹ, אִם יִהְיוּ כָל חַכְמֵי יִשְׂרָאֵל בְּכַף מֹאזְנַיִם וְרַבִּי אֱלִיעֶזֶר בֶּן הוֹרְקְנוֹס אַף עִמָּהֶם, וְרַבִּי אֶלְעָזָר בֶּן עֲרָךְ בְּכַף שְׁנִיָּה, מַכְרִיעַ אֶת כֻּלָּם:
אָמַר לָהֶם, צְאוּ וּרְאוּ אֵיזוֹהִי דֶרֶךְ יְשָׁרָה שֶׁיִּדְבַּק בָּהּ הָאָדָם. רַבִּי אֱלִיעֶזֶר אוֹמֵר, עַיִן טוֹבָה. רַבִּי יְהוֹשֻׁעַ אוֹמֵר, חָבֵר טוֹב. רַבִּי יוֹסֵי אוֹמֵר, שָׁכֵן טוֹב. רַבִּי שִׁמְעוֹן אוֹמֵר, הָרוֹאֶה אֶת הַנּוֹלָד. רַבִּי אֶלְעָזָר אוֹמֵר, לֵב טוֹב. אָמַר לָהֶם, רוֹאֶה אֲנִי אֶת דִּבְרֵי אֶלְעָזָר בֶּן עֲרָךְ מִדִּבְרֵיכֶם, שֶׁבִּכְלָל דְּבָרָיו דִּבְרֵיכֶם. אָמַר לָהֶם צְאוּ וּרְאוּ אֵיזוֹהִי דֶרֶךְ רָעָה שֶׁיִּתְרַחֵק מִמֶּנָּה הָאָדָם. רַבִּי אֱלִיעֶזֶר אוֹמֵר, עַיִן רָעָה. רַבִּי יְהוֹשֻׁעַ אוֹמֵר, חָבֵר רָע. רַבִּי יוֹסֵי אוֹמֵר, שָׁכֵן רָע. רַבִּי שִׁמְעוֹן אוֹמֵר, הַלֹּוֶה וְאֵינוֹ מְשַׁלֵּם. אֶחָד הַלֹּוֶה מִן הָאָדָם, כְּלֹוֶה מִן הַמָּקוֹם בָּרוּךְ הוּא, שֶׁנֶּאֱמַר (תהלים לז) לֹוֶה רָשָׁע וְלֹא יְשַׁלֵּם, וְצַדִּיק חוֹנֵן וְנוֹתֵן. רַבִּי אֶלְעָזָר אוֹמֵר, לֵב רָע. אָמַר לָהֶם, רוֹאֶה אֲנִי אֶת דִּבְרֵי אֶלְעָזָר בֶּן עֲרָךְ מִדִּבְרֵיכֶם, שֶׁבִּכְלָל דְּבָרָיו דִּבְרֵיכֶם:
הֵם אָמְרוּ שְׁלשָׁה דְבָרִים. רַבִּי אֱלִיעֶזֶר אוֹמֵר, יְהִי כְבוֹד חֲבֵרְךָ חָבִיב עָלֶיךָ כְּשֶׁלָּךְ, וְאַל תְּהִי נוֹחַ לִכְעֹס. וְשׁוּב יוֹם אֶחָד לִפְנֵי מִיתָתְךָ. וֶהֱוֵי מִתְחַמֵּם כְּנֶגֶד אוּרָן שֶׁל חֲכָמִים, וֶהֱוֵי זָהִיר בְּגַחַלְתָּן שֶׁלֹּא תִכָּוֶה, שֶׁנְּשִׁיכָתָן נְשִׁיכַת שׁוּעָל, וַעֲקִיצָתָן עֲקִיצַת עַקְרָב, וּלְחִישָׁתָן לְחִישַׁת שָׂרָף, וְכָל דִּבְרֵיהֶם כְּגַחֲלֵי אֵשׁ:
רַבִּי יְהוֹשֻׁעַ אוֹמֵר, עַיִן הָרָע, וְיֵצֶר הָרָע, וְשִׂנְאַת הַבְּרִיּוֹת, מוֹצִיאִין אֶת הָאָדָם מִן הָעוֹלָם:
רַבִּי יוֹסֵי אוֹמֵר, יְהִי מָמוֹן חֲבֵרְךָ חָבִיב עָלֶיךָ כְּשֶׁלָּךְ, וְהַתְקֵן עַצְמְךָ לִלְמֹד תּוֹרָה, שֶׁאֵינָהּ יְרֻשָּׁה לָךְ. וְכָל מַעֲשֶׂיךָ יִהְיוּ לְשֵׁם שָׁמָיִם:
רַבִּי שִׁמְעוֹן אוֹמֵר, הֱוֵי זָהִיר בִּקְרִיאַת שְׁמַע וּבַתְּפִלָּה. וּכְשֶׁאַתָּה מִתְפַּלֵּל, אַל תַּעַשׂ תְּפִלָּתְךָ קֶבַע, אֶלָּא רַחֲמִים וְתַחֲנוּנִים לִפְנֵי הַמָּקוֹם בָּרוּךְ הוּא, שֶׁנֶּאֱמַר (יואל ב) כִּי חַנּוּן וְרַחוּם הוּא אֶרֶךְ אַפַּיִם וְרַב חֶסֶד וְנִחָם עַל הָרָעָה. וְאַל תְּהִי רָשָׁע בִּפְנֵי עַצְמְךָ:
רַבִּי אֶלְעָזָר אוֹמֵר, הֱוֵי שָׁקוּד לִלְמֹד תּוֹרָה, וְדַע מַה שֶּׁתָּשִׁיב לְאֶפִּיקוֹרוֹס. וְדַע לִפְנֵי מִי אַתָּה עָמֵל. וְנֶאֱמָן הוּא בַעַל מְלַאכְתְּךָ שֶׁיְּשַׁלֶּם לָךְ שְׂכַר פְּעֻלָּתֶךְ:
רַבִּי טַרְפוֹן אוֹמֵר, הַיּוֹם קָצָר וְהַמְּלָאכָה מְרֻבָּה, וְהַפּוֹעֲלִים עֲצֵלִים, וְהַשָּׂכָר הַרְבֵּה, וּבַעַל הַבַּיִת דּוֹחֵק:
הוּא הָיָה אוֹמֵר, לֹא עָלֶיךָ הַמְּלָאכָה לִגְמֹר, וְלֹא אַתָּה בֶן חוֹרִין לִבָּטֵל מִמֶּנָּה. אִם לָמַדְתָּ תוֹרָה הַרְבֵּה, נוֹתְנִים לְךָ שָׂכָר הַרְבֵּה. וְנֶאֱמָן הוּא בַעַל מְלַאכְתְּךָ שֶׁיְּשַׁלֵּם לְךָ שְׂכַר פְּעֻלָּתֶךָ. וְדַע מַתַּן שְׂכָרָן שֶׁל צַדִּיקִים לֶעָתִיד לָבֹא:

Chapter 3

עֲקַבְיָא בֶן מַהֲלַלְאֵל אוֹמֵר, הִסְתַּכֵּל בִּשְׁלשָׁה דְבָרִים וְאִי אַתָּה בָא לִידֵי עֲבֵרָה. דַּע מֵאַיִן בָּאתָ, וּלְאָן אַתָּה הוֹלֵךְ, וְלִפְנֵי מִי אַתָּה עָתִיד לִתֵּן דִּין וְחֶשְׁבּוֹן. מֵאַיִן בָּאתָ, מִטִּפָּה סְרוּחָה, וּלְאָן אַתָּה הוֹלֵךְ, לִמְקוֹם עָפָר רִמָּה וְתוֹלֵעָה. וְלִפְנֵי מִי אַתָּה עָתִיד לִתֵּן דִּין וְחֶשְׁבּוֹן, לִפְנֵי מֶלֶךְ מַלְכֵי הַמְּלָכִים הַקָּדוֹשׁ בָּרוּךְ הוּא:
רַבִּי חֲנִינָא סְגַן הַכֹּהֲנִים אוֹמֵר, הֱוֵי מִתְפַּלֵּל בִּשְׁלוֹמָהּ שֶׁל מַלְכוּת, שֶׁאִלְמָלֵא מוֹרָאָהּ, אִישׁ אֶת רֵעֵהוּ חַיִּים בְּלָעוֹ. רַבִּי חֲנִינָא בֶן תְּרַדְיוֹן אוֹמֵר, שְׁנַיִם שֶׁיּוֹשְׁבִין וְאֵין בֵּינֵיהֶן דִּבְרֵי תוֹרָה, הֲרֵי זֶה מוֹשַׁב לֵצִים, שֶׁנֶּאֱמַר (תהלים א) וּבְמוֹשַׁב לֵצִים לֹא יָשָׁב. אֲבָל שְׁנַיִם שֶׁיּוֹשְׁבִין וְיֵשׁ בֵּינֵיהֶם דִּבְרֵי תוֹרָה, שְׁכִינָה שְׁרוּיָה בֵינֵיהֶם, שֶׁנֶּאֱמַר (מלאכי ג) אָז נִדְבְּרוּ יִרְאֵי יְיָ אִישׁ אֶל רֵעֵהוּ וַיַּקְשֵׁב יְיָ וַיִּשְׁמָע וַיִּכָּתֵב סֵפֶר זִכָּרוֹן לְפָנָיו לְיִרְאֵי יְיָ וּלְחֹשְׁבֵי שְׁמוֹ. אֵין לִי אֶלָּא שְׁנַיִם, מִנַּיִן שֶׁאֲפִלּוּ אֶחָד שֶׁיּוֹשֵׁב וְעוֹסֵק בַּתּוֹרָה, שֶׁהַקָּדוֹשׁ בָּרוּךְ הוּא קוֹבֵעַ לוֹ שָׂכָר, שֶׁנֶּאֱמַר (איכה ג) יֵשֵׁב בָּדָד וְיִדֹּם כִּי נָטַל עָלָיו:
רַבִּי שִׁמְעוֹן אוֹמֵר, שְׁלשָׁה שֶׁאָכְלוּ עַל שֻׁלְחָן אֶחָד וְלֹא אָמְרוּ עָלָיו דִּבְרֵי תוֹרָה, כְּאִלּוּ אָכְלוּ מִזִּבְחֵי מֵתִים, שֶׁנֶּאֱמַר (ישעיה כח) כִּי כָּל שֻׁלְחָנוֹת מָלְאוּ קִיא צֹאָה בְּלִי מָקוֹם. אֲבָל שְׁלשָׁה שֶׁאָכְלוּ עַל שֻׁלְחָן אֶחָד וְאָמְרוּ עָלָיו דִּבְרֵי תוֹרָה, כְּאִלּוּ אָכְלוּ מִשֻּׁלְחָנוֹ שֶׁל מָקוֹם בָּרוּךְ הוּא, שֶׁנֶּאֱמַר (יחזקאל מא) וַיְדַבֵּר אֵלַי זֶה הַשֻּׁלְחָן אֲשֶׁר לִפְנֵי ה':
רַבִּי חֲנִינָא בֶן חֲכִינַאי אוֹמֵר, הַנֵּעוֹר בַּלַּיְלָה וְהַמְהַלֵּךְ בַּדֶּרֶךְ יְחִידִי וְהַמְפַנֶּה לִבּוֹ לְבַטָּלָה, הֲרֵי זֶה מִתְחַיֵּב בְּנַפְשׁוֹ:
רַבִּי נְחוּנְיָא בֶּן הַקָּנָה אוֹמֵר, כָּל הַמְקַבֵּל עָלָיו עֹל תּוֹרָה, מַעֲבִירִין מִמֶּנּוּ עֹל מַלְכוּת וְעֹל דֶּרֶךְ אֶרֶץ. וְכָל הַפּוֹרֵק מִמֶּנּוּ עֹל תּוֹרָה, נוֹתְנִין עָלָיו עֹל מַלְכוּת וְעֹל דֶּרֶךְ אֶרֶץ:
רַבִּי חֲלַפְתָּא בֶן דּוֹסָא אִישׁ כְּפַר חֲנַנְיָה אוֹמֵר, עֲשָׂרָה שֶׁיּוֹשְׁבִין וְעוֹסְקִין בַּתּוֹרָה, שְׁכִינָה שְׁרוּיָה בֵינֵיהֶם, שֶׁנֶּאֱמַר (תהלים פב) אֱלֹהִים נִצָּב בַּעֲדַת אֵל. וּמִנַּיִן אֲפִלּוּ חֲמִשָּׁה, שֶׁנֶּאֱמַר (עמוס ט) וַאֲגֻדָּתוֹ עַל אֶרֶץ יְסָדָהּ. וּמִנַּיִן אֲפִלּוּ שְׁלשָׁה, שֶׁנֶּאֱמַר (תהלים פב) בְּקֶרֶב אֱלֹהִים יִשְׁפֹּט. וּמִנַּיִן אֲפִלּוּ שְׁנַיִם, שֶׁנֶּאֱמַר (מלאכי ג) אָז נִדְבְּרוּ יִרְאֵי ה' אִישׁ אֶל רֵעֵהוּ וַיַּקְשֵׁב ה' וַיִּשְׁמָע וְגוֹ'. וּמִנַּיִן אֲפִלּוּ אֶחָד, שֶׁנֶּאֱמַר (שמות כ) בְּכָל הַמָּקוֹם אֲשֶׁר אַזְכִּיר אֶת שְׁמִי אָבֹא אֵלֶיךָ וּבֵרַכְתִּיךָ:
רַבִּי אֶלְעָזָר אִישׁ בַּרְתּוֹתָא אוֹמֵר, תֶּן לוֹ מִשֶּׁלּוֹ, שֶׁאַתָּה וְשֶׁלְּךָ שֶׁלּוֹ. וְכֵן בְּדָוִד הוּא אוֹמֵר (דברי הימים א כט) כִּי מִמְּךָ הַכֹּל וּמִיָּדְךָ נָתַנּוּ לָךְ. רַבִּי שִׁמְעוֹן אוֹמֵר, הַמְהַלֵּךְ בַּדֶּרֶךְ וְשׁוֹנֶה, וּמַפְסִיק מִמִּשְׁנָתוֹ וְאוֹמֵר, מַה נָּאֶה אִילָן זֶה וּמַה נָּאֶה נִיר זֶה, מַעֲלֶה עָלָיו הַכָּתוּב כְּאִלּוּ מִתְחַיֵּב בְּנַפְשׁוֹ:
רַבִּי דּוֹסְתַּאי בְּרַבִּי יַנַּאי מִשּׁוּם רַבִּי מֵאִיר אוֹמֵר, כָּל הַשּׁוֹכֵחַ דָּבָר אֶחָד מִמִּשְׁנָתוֹ, מַעֲלֶה עָלָיו הַכָּתוּב כְּאִלּוּ מִתְחַיֵּב בְּנַפְשׁוֹ, שֶׁנֶּאֱמַר (דברים ד) רַק הִשָּׁמֶר לְךָ וּשְׁמֹר נַפְשְׁךָ מְאֹד פֶּן תִּשְׁכַּח אֶת הַדְּבָרִים אֲשֶׁר רָאוּ עֵינֶיךָ. יָכוֹל אֲפִלּוּ תָקְפָה עָלָיו מִשְׁנָתוֹ, תַּלְמוּד לוֹמַר (שם) וּפֶן יָסוּרוּ מִלְּבָבְךָ כֹּל יְמֵי חַיֶּיךָ, הָא אֵינוֹ מִתְחַיֵּב בְּנַפְשׁוֹ עַד שֶׁיֵּשֵׁב וִיסִירֵם מִלִּבּוֹ:
רַבִּי חֲנִינָא בֶן דּוֹסָא אוֹמֵר, כָּל שֶׁיִּרְאַת חֶטְאוֹ קוֹדֶמֶת לְחָכְמָתוֹ, חָכְמָתוֹ מִתְקַיֶּמֶת. וְכָל שֶׁחָכְמָתוֹ קוֹדֶמֶת לְיִרְאַת חֶטְאוֹ, אֵין חָכְמָתוֹ מִתְקַיֶּמֶת. הוּא הָיָה אוֹמֵר, כָּל שֶׁמַּעֲשָׂיו מְרֻבִּין מֵחָכְמָתוֹ, חָכְמָתוֹ מִתְקַיֶּמֶת. וְכָל שֶׁחָכְמָתוֹ מְרֻבָּה מִמַּעֲשָׂיו, אֵין חָכְמָתוֹ מִתְקַיֶּמֶת:
הוּא הָיָה אוֹמֵר, כָּל שֶׁרוּחַ הַבְּרִיּוֹת נוֹחָה הֵימֶנּוּ, רוּחַ הַמָּקוֹם נוֹחָה הֵימֶנּוּ. וְכָל שֶׁאֵין רוּחַ הַבְּרִיּוֹת נוֹחָה הֵימֶנּוּ, אֵין רוּחַ הַמָּקוֹם נוֹחָה הֵימֶנּוּ. רַבִּי דוֹסָא בֶן הַרְכִּינַס אוֹמֵר, שֵׁנָה שֶׁל שַׁחֲרִית, וְיַיִן שֶׁל צָהֳרַיִם, וְשִׂיחַת הַיְלָדִים, וִישִׁיבַת בָּתֵּי כְנֵסִיּוֹת שֶׁל עַמֵּי הָאָרֶץ, מוֹצִיאִין אֶת הָאָדָם מִן הָעוֹלָם:
רַבִּי אֶלְעָזָר הַמּוֹדָעִי אוֹמֵר, הַמְחַלֵּל אֶת הַקָּדָשִׁים, וְהַמְבַזֶּה אֶת הַמּוֹעֲדוֹת, וְהַמַּלְבִּין פְּנֵי חֲבֵרוֹ בָרַבִּים, וְהַמֵּפֵר בְּרִיתוֹ שֶׁל אַבְרָהָם אָבִינוּ עָלָיו הַשָּׁלוֹם, וְהַמְגַלֶּה פָנִים בַּתּוֹרָה שֶׁלֹּא כַהֲלָכָה, אַף עַל פִּי שֶׁיֵּשׁ בְּיָדוֹ תוֹרָה וּמַעֲשִׂים טוֹבִים, אֵין לוֹ חֵלֶק לָעוֹלָם הַבָּא:
רַבִּי יִשְׁמָעֵאל אוֹמֵר, הֱוֵי קַל לְרֹאשׁ, וְנוֹחַ לְתִשְׁחֹרֶת, וֶהֱוֵי מְקַבֵּל אֶת כָּל הָאָדָם בְּשִׂמְחָה:
רַבִּי עֲקִיבָא אוֹמֵר, שְׂחוֹק וְקַלּוּת רֹאשׁ, מַרְגִּילִין לְעֶרְוָה. מָסֹרֶת, סְיָג לַתּוֹרָה. מַעַשְׂרוֹת, סְיָג לָעשֶׁר. נְדָרִים, סְיָג לַפְּרִישׁוּת. סְיָג לַחָכְמָה, שְׁתִיקָה:
הוּא הָיָה אוֹמֵר, חָבִיב אָדָם שֶׁנִּבְרָא בְצֶלֶם. חִבָּה יְתֵרָה נוֹדַעַת לוֹ שֶׁנִּבְרָא בְצֶלֶם, שֶׁנֶּאֱמַר (בראשית ט) כִּי בְּצֶלֶם אֱלֹהִים עָשָׂה אֶת הָאָדָם. חֲבִיבִין יִשְׂרָאֵל שֶׁנִּקְרְאוּ בָנִים לַמָּקוֹם. חִבָּה יְתֵרָה נוֹדַעַת לָהֶם שֶׁנִּקְרְאוּ בָנִים לַמָּקוֹם, שֶׁנֶּאֱמַר (דברים יד) בָּנִים אַתֶּם לַה' אֱלֹהֵיכֶם. חֲבִיבִין יִשְׂרָאֵל שֶׁנִּתַּן לָהֶם כְּלִי חֶמְדָּה. חִבָּה יְתֵרָה נוֹדַעַת לָהֶם שֶׁנִּתַּן לָהֶם כְּלִי חֶמְדָּה שֶׁבּוֹ נִבְרָא הָעוֹלָם, שֶׁנֶּאֱמַר (משלי ד) כִּי לֶקַח טוֹב נָתַתִּי לָכֶם, תּוֹרָתִי אַל תַּעֲזֹבוּ:
הַכֹּל צָפוּי, וְהָרְשׁוּת נְתוּנָה, וּבְטוֹב הָעוֹלָם נִדּוֹן. וְהַכֹּל לְפִי רֹב הַמַּעֲשֶׂה:
הוּא הָיָה אוֹמֵר, הַכֹּל נָתוּן בְּעֵרָבוֹן, וּמְצוּדָה פְרוּסָה עַל כָּל הַחַיִּים. הַחֲנוּת פְּתוּחָה, וְהַחֶנְוָנִי מֵקִיף, וְהַפִּנְקָס פָּתוּחַ, וְהַיָּד כּוֹתֶבֶת, וְכָל הָרוֹצֶה לִלְווֹת יָבֹא וְיִלְוֶה, וְהַגַּבָּאִים מַחֲזִירִים תָּדִיר בְּכָל יוֹם, וְנִפְרָעִין מִן הָאָדָם מִדַּעְתּוֹ וְשֶׁלֹּא מִדַּעְתּוֹ, וְיֵשׁ לָהֶם עַל מַה שֶּׁיִּסְמֹכוּ, וְהַדִּין דִּין אֱמֶת, וְהַכֹּל מְתֻקָּן לַסְּעוּדָה:
רַבִּי אֶלְעָזָר בֶּן עֲזַרְיָה אוֹמֵר, אִם אֵין תּוֹרָה, אֵין דֶּרֶךְ אֶרֶץ. אִם אֵין דֶּרֶךְ אֶרֶץ, אֵין תּוֹרָה. אִם אֵין חָכְמָה, אֵין יִרְאָה. אִם אֵין יִרְאָה, אֵין חָכְמָה. אִם אֵין בִּינָה, אֵין דַּעַת. אִם אֵין דַּעַת, אֵין בִּינָה. אִם אֵין קֶמַח, אֵין תּוֹרָה. אִם אֵין תּוֹרָה, אֵין קֶמַח. הוּא הָיָה אוֹמֵר, כָּל שֶׁחָכְמָתוֹ מְרֻבָּה מִמַּעֲשָׂיו, לְמַה הוּא דוֹמֶה, לְאִילָן שֶׁעֲנָפָיו מְרֻבִּין וְשָׁרָשָׁיו מֻעָטִין, וְהָרוּחַ בָּאָה וְעוֹקַרְתּוֹ וְהוֹפַכְתּוֹ עַל פָּנָיו, שֶׁנֶּאֱמַר (ירמיה יז) וְהָיָה כְּעַרְעָר בָּעֲרָבָה וְלֹא יִרְאֶה כִּי יָבוֹא טוֹב וְשָׁכַן חֲרֵרִים בַּמִּדְבָּר אֶרֶץ מְלֵחָה וְלֹא תֵשֵׁב. אֲבָל כָּל שֶׁמַּעֲשָׂיו מְרֻבִּין מֵחָכְמָתוֹ, לְמַה הוּא דוֹמֶה, לְאִילָן שֶׁעֲנָפָיו מֻעָטִין וְשָׁרָשָׁיו מְרֻבִּין, שֶׁאֲפִלּוּ כָל הָרוּחוֹת שֶׁבָּעוֹלָם בָּאוֹת וְנוֹשְׁבוֹת בּוֹ אֵין מְזִיזִין אוֹתוֹ מִמְּקוֹמוֹ, שֶׁנֶּאֱמַר (שם) וְהָיָה כְּעֵץ שָׁתוּל עַל מַיִם וְעַל יוּבַל יְשַׁלַּח שָׁרָשָׁיו וְלֹא יִרְאֶה כִּי יָבֹא חֹם, וְהָיָה עָלֵהוּ רַעֲנָן, וּבִשְׁנַת בַּצֹּרֶת לֹא יִדְאָג, וְלֹא יָמִישׁ מֵעֲשׂוֹת פֶּרִי:
רַבִּי אֱלִיעֶזֶר בֶּן חִסְמָא אוֹמֵר, קִנִּין וּפִתְחֵי נִדָּה, הֵן הֵן גּוּפֵי הֲלָכוֹת. תְּקוּפוֹת וְגִימַטְרִיאוֹת, פַּרְפְּרָאוֹת לַחָכְמָה:

Chapter 4

בֶּן זוֹמָא אוֹמֵר, אֵיזֶהוּ חָכָם, הַלּוֹמֵד מִכָּל אָדָם, שֶׁנֶּאֱמַר (תהלים קיט) מִכָּל מְלַמְּדַי הִשְׂכַּלְתִּי כִּי עֵדְוֹתֶיךָ שִׂיחָה לִּי. אֵיזֶהוּ גִבּוֹר, הַכּוֹבֵשׁ אֶת יִצְרוֹ, שֶׁנֶּאֱמַר (משלי טז) טוֹב אֶרֶךְ אַפַּיִם מִגִּבּוֹר וּמשֵׁל בְּרוּחוֹ מִלֹּכֵד עִיר. אֵיזֶהוּ עָשִׁיר, הַשָּׂמֵחַ בְּחֶלְקוֹ, שֶׁנֶּאֱמַר (תהלים קכח) יְגִיעַ כַּפֶּיךָ כִּי תֹאכֵל אַשְׁרֶיךָ וְטוֹב לָךְ. אַשְׁרֶיךָ, בָּעוֹלָם הַזֶּה. וְטוֹב לָךְ, לָעוֹלָם הַבָּא. אֵיזֶהוּ מְכֻבָּד, הַמְכַבֵּד אֶת הַבְּרִיּוֹת, שֶׁנֶּאֱמַר (שמואל א ב) כִּי מְכַבְּדַי אֲכַבֵּד וּבֹזַי יֵקָלּוּ:
בֶּן עַזַּאי אוֹמֵר, הֱוֵי רָץ לְמִצְוָה קַלָּה כְבַחֲמוּרָה, וּבוֹרֵחַ מִן הָעֲבֵרָה. שֶׁמִּצְוָה גּוֹרֶרֶת מִצְוָה, וַעֲבֵרָה גוֹרֶרֶת עֲבֵרָה. שֶׁשְּׂכַר מִצְוָה, מִצְוָה. וּשְׂכַר עֲבֵרָה, עֲבֵרָה:
הוּא הָיָה אוֹמֵר, אַל תְּהִי בָז לְכָל אָדָם, וְאַל תְּהִי מַפְלִיג לְכָל דָּבָר, שֶׁאֵין לְךָ אָדָם שֶׁאֵין לוֹ שָׁעָה וְאֵין לְךָ דָבָר שֶׁאֵין לוֹ מָקוֹם:
רַבִּי לְוִיטָס אִישׁ יַבְנֶה אוֹמֵר, מְאֹד מְאֹד הֱוֵי שְׁפַל רוּחַ, שֶׁתִּקְוַת אֱנוֹשׁ רִמָּה. רַבִּי יוֹחָנָן בֶּן בְּרוֹקָא אוֹמֵר, כָּל הַמְחַלֵּל שֵׁם שָׁמַיִם בַּסֵּתֶר, נִפְרָעִין מִמֶּנּוּ בְגָלוּי. אֶחָד שׁוֹגֵג וְאֶחָד מֵזִיד בְּחִלּוּל הַשֵּׁם:
רַבִּי יִשְׁמָעֵאל בְּנוֹ אוֹמֵר, הַלּוֹמֵד תּוֹרָה עַל מְנָת לְלַמֵּד, מַסְפִּיקִין בְּיָדוֹ לִלְמֹד וּלְלַמֵּד. וְהַלּוֹמֵד עַל מְנָת לַעֲשׂוֹת, מַסְפִּיקִין בְּיָדוֹ לִלְמֹד וּלְלַמֵּד לִשְׁמֹר וְלַעֲשׂוֹת. רַבִּי צָדוֹק אוֹמֵר, אַל תַּעֲשֵׂם עֲטָרָה לְהִתְגַּדֵּל בָּהֶם, וְלֹא קַרְדֹּם לַחְפֹּר בָּהֶם. וְכָךְ הָיָה הִלֵּל אוֹמֵר, וּדְאִשְׁתַּמֵּשׁ בְּתָגָא, חָלָף. הָא לָמַדְתָּ, כָּל הַנֶּהֱנֶה מִדִּבְרֵי תוֹרָה, נוֹטֵל חַיָּיו מִן הָעוֹלָם:
רַבִּי יוֹסֵי אוֹמֵר, כָּל הַמְכַבֵּד אֶת הַתּוֹרָה, גּוּפוֹ מְכֻבָּד עַל הַבְּרִיּוֹת. וְכָל הַמְחַלֵּל אֶת הַתּוֹרָה, גּוּפוֹ מְחֻלָּל עַל הַבְּרִיּוֹת:
רַבִּי יִשְׁמָעֵאל בְּנוֹ אוֹמֵר, הַחוֹשֵׂךְ עַצְמוֹ מִן הַדִּין, פּוֹרֵק מִמֶּנּוּ אֵיבָה וְגָזֵל וּשְׁבוּעַת שָׁוְא. וְהַגַּס לִבּוֹ בַהוֹרָאָה, שׁוֹטֶה רָשָׁע וְגַס רוּחַ:
הוּא הָיָה אוֹמֵר, אַל תְּהִי דָן יְחִידִי, שֶׁאֵין דָּן יְחִידִי אֶלָּא אֶחָד. וְאַל תֹּאמַר קַבְּלוּ דַעְתִּי, שֶׁהֵן רַשָּׁאִין וְלֹא אָתָּה:
רַבִּי יוֹנָתָן אוֹמֵר, כָּל הַמְקַיֵּם אֶת הַתּוֹרָה מֵעֹנִי, סוֹפוֹ לְקַיְּמָהּ מֵעשֶׁר. וְכָל הַמְבַטֵּל אֶת הַתּוֹרָה מֵעשֶׁר, סוֹפוֹ לְבַטְּלָהּ מֵעֹנִי:
רַבִּי מֵאִיר אוֹמֵר, הֱוֵי מְמַעֵט בְּעֵסֶק, וַעֲסֹק בַּתּוֹרָה. וֶהֱוֵי שְׁפַל רוּחַ בִּפְנֵי כָל אָדָם. וְאִם בָּטַלְתָּ מִן הַתּוֹרָה, יֶשׁ לְךָ בְטֵלִים הַרְבֵּה כְנֶגְדָּךְ. וְאִם עָמַלְתָּ בַתּוֹרָה, יֶשׁ לוֹ שָׂכָר הַרְבֵּה לִתֶּן לָךְ:
רַבִּי אֱלִיעֶזֶר בֶּן יַעֲקֹב אוֹמֵר, הָעוֹשֶׂה מִצְוָה אַחַת, קוֹנֶה לוֹ פְרַקְלִיט אֶחָד. וְהָעוֹבֵר עֲבֵרָה אַחַת, קוֹנֶה לוֹ קַטֵּגוֹר אֶחָד. תְּשׁוּבָה וּמַעֲשִׂים טוֹבִים, כִּתְרִיס בִּפְנֵי הַפֻּרְעָנוּת. רַבִּי יוֹחָנָן הַסַּנְדְּלָר אוֹמֵר, כָּל כְּנֵסִיָּה שֶׁהִיא לְשֵׁם שָׁמַיִם, סוֹפָהּ לְהִתְקַיֵּם. וְשֶׁאֵינָהּ לְשֵׁם שָׁמַיִם, אֵין סוֹפָהּ לְהִתְקַיֵּם:
רַבִּי אֶלְעָזָר בֶּן שַׁמּוּעַ אוֹמֵר, יְהִי כְבוֹד תַּלְמִידְךָ חָבִיב עָלֶיךָ כְּשֶׁלְּךָ, וּכְבוֹד חֲבֵרְךָ כְּמוֹרָא רַבְּךָ, וּמוֹרָא רַבְּךָ כְּמוֹרָא שָׁמָיִם:
רַבִּי יְהוּדָה אוֹמֵר, הֱוֵי זָהִיר בַּתַּלְמוּד, שֶׁשִּׁגְגַת תַּלְמוּד עוֹלָה זָדוֹן. רַבִּי שִׁמְעוֹן אוֹמֵר, שְׁלשָׁה כְתָרִים הֵם, כֶּתֶר תּוֹרָה וְכֶתֶר כְּהֻנָּה וְכֶתֶר מַלְכוּת, וְכֶתֶר שֵׁם טוֹב עוֹלֶה עַל גַּבֵּיהֶן:
רַבִּי נְהוֹרַאי אוֹמֵר, הֱוֵי גוֹלֶה לִמְקוֹם תּוֹרָה, וְאַל תֹּאמַר שֶׁהִיא תָבֹא אַחֲרֶיךָ, שֶׁחֲבֵרֶיךָ יְקַיְּמוּהָ בְיָדֶךָ. וְאֶל בִּינָתְךָ אַל תִּשָּׁעֵן (משלי ג):
רַבִּי יַנַּאי אוֹמֵר, אֵין בְּיָדֵינוּ לֹא מִשַּׁלְוַת הָרְשָׁעִים וְאַף לֹא מִיִּסּוּרֵי הַצַּדִּיקִים. רַבִּי מַתְיָא בֶן חָרָשׁ אוֹמֵר, הֱוֵי מַקְדִּים בִּשְׁלוֹם כָּל אָדָם. וֶהֱוֵי זָנָב לָאֲרָיוֹת, וְאַל תְּהִי רֹאשׁ לַשּׁוּעָלִים:
רַבִּי יַעֲקֹב אוֹמֵר, הָעוֹלָם הַזֶּה דּוֹמֶה לִפְרוֹזְדוֹר בִּפְנֵי הָעוֹלָם הַבָּא. הַתְקֵן עַצְמְךָ בַפְּרוֹזְדוֹר, כְּדֵי שֶׁתִּכָּנֵס לַטְּרַקְלִין:
הוּא הָיָה אוֹמֵר, יָפָה שָׁעָה אַחַת בִּתְשׁוּבָה וּמַעֲשִׂים טוֹבִים בָּעוֹלָם הַזֶּה, מִכָּל חַיֵּי הָעוֹלָם הַבָּא. וְיָפָה שָׁעָה אַחַת שֶׁל קוֹרַת רוּחַ בָּעוֹלָם הַבָּא, מִכָּל חַיֵּי הָעוֹלָם הַזֶּה:
רַבִּי שִׁמְעוֹן בֶּן אֶלְעָזָר אוֹמֵר, אַל תְּרַצֶּה אֶת חֲבֵרְךָ בִשְׁעַת כַּעֲסוֹ, וְאַל תְּנַחֲמֶנּוּ בְּשָׁעָה שֶׁמֵּתוֹ מֻטָּל לְפָנָיו, וְאַל תִּשְׁאַל לוֹ בִשְׁעַת נִדְרוֹ, וְאַל תִּשְׁתַּדֵּל לִרְאוֹתוֹ בִשְׁעַת קַלְקָלָתוֹ:
שְׁמוּאֵל הַקָּטָן אוֹמֵר, (משלי כד) בִּנְפֹל אוֹיִבְךָ אַל תִּשְׂמָח וּבִכָּשְׁלוֹ אַל יָגֵל לִבֶּךָ, פֶּן יִרְאֶה ה' וְרַע בְּעֵינָיו וְהֵשִׁיב מֵעָלָיו אַפּוֹ:
אֱלִישָׁע בֶּן אֲבוּיָה אוֹמֵר, הַלּוֹמֵד יֶלֶד לְמַה הוּא דוֹמֶה, לִדְיוֹ כְתוּבָה עַל נְיָר חָדָשׁ. וְהַלּוֹמֵד זָקֵן לְמַה הוּא דוֹמֶה, לִדְיוֹ כְתוּבָה עַל נְיָר מָחוּק. רַבִּי יוֹסֵי בַר יְהוּדָה אִישׁ כְּפַר הַבַּבְלִי אוֹמֵר, הַלּוֹמֵד מִן הַקְּטַנִּים לְמַה הוּא דוֹמֶה, לְאֹכֵל עֲנָבִים קֵהוֹת וְשׁוֹתֶה יַיִן מִגִּתּוֹ. וְהַלּוֹמֵד מִן הַזְּקֵנִים לְמַה הוּא דוֹמֶה, לְאֹכֵל עֲנָבִים בְּשֵׁלוֹת וְשׁוֹתֶה יַיִן יָשָׁן. רַבִּי אוֹמֵר, אַל תִּסְתַּכֵּל בַּקַּנְקַן, אֶלָּא בְמַה שֶּׁיֶּשׁ בּוֹ. יֵשׁ קַנְקַן חָדָשׁ מָלֵא יָשָׁן, וְיָשָׁן שֶׁאֲפִלּוּ חָדָשׁ אֵין בּוֹ:
רַבִּי אֶלְעָזָר הַקַּפָּר אוֹמֵר, הַקִּנְאָה וְהַתַּאֲוָה וְהַכָּבוֹד, מוֹצִיאִין אֶת הָאָדָם מִן הָעוֹלָם:
הוּא הָיָה אוֹמֵר, הַיִּלּוֹדִים לָמוּת, וְהַמֵּתִים לְהֵחָיוֹת, וְהַחַיִּים לִדּוֹן. לֵידַע לְהוֹדִיעַ וּלְהִוָּדַע שֶׁהוּא אֵל, הוּא הַיּוֹצֵר, הוּא הַבּוֹרֵא, הוּא הַמֵּבִין, הוּא הַדַּיָּן, הוּא עֵד, הוּא בַעַל דִּין, וְהוּא עָתִיד לָדוּן. בָּרוּךְ הוּא, שֶׁאֵין לְפָנָיו לֹא עַוְלָה, וְלֹא שִׁכְחָה, וְלֹא מַשּׂוֹא פָנִים, וְלֹא מִקַּח שֹׁחַד, שֶׁהַכֹּל שֶׁלּוֹ. וְדַע שֶׁהַכֹּל לְפִי הַחֶשְׁבּוֹן. וְאַל יַבְטִיחֲךָ יִצְרְךָ שֶׁהַשְּׁאוֹל בֵּית מָנוֹס לְךָ, שֶׁעַל כָּרְחֲךָ אַתָּה נוֹצָר, וְעַל כָּרְחֲךָ אַתָּה נוֹלָד, וְעַל כָּרְחֲךָ אַתָּה חַי, וְעַל כָּרְחֲךָ אַתָּה מֵת, וְעַל כָּרְחֲךָ אַתָּה עָתִיד לִתֵּן דִּין וְחֶשְׁבּוֹן לִפְנֵי מֶלֶךְ מַלְכֵי הַמְּלָכִים הַקָּדוֹשׁ בָּרוּךְ הוּא:

Chapter 5

בַּעֲשָׂרָה מַאֲמָרוֹת נִבְרָא הָעוֹלָם. וּמַה תַּלְמוּד לוֹמַר, וַהֲלֹא בְמַאֲמָר אֶחָד יָכוֹל לְהִבָּרְאוֹת, אֶלָּא לְהִפָּרַע מִן הָרְשָׁעִים שֶׁמְּאַבְּדִין אֶת הָעוֹלָם שֶׁנִּבְרָא בַעֲשָׂרָה מַאֲמָרוֹת, וְלִתֵּן שָׂכָר טוֹב לַצַּדִּיקִים שֶׁמְּקַיְּמִין אֶת הָעוֹלָם שֶׁנִּבְרָא בַעֲשָׂרָה מַאֲמָרוֹת:
עֲשָׂרָה דוֹרוֹת מֵאָדָם וְעַד נֹחַ, לְהוֹדִיעַ כַּמָּה אֶרֶךְ אַפַּיִם לְפָנָיו, שֶׁכָּל הַדּוֹרוֹת הָיוּ מַכְעִיסִין וּבָאִין עַד שֶׁהֵבִיא עֲלֵיהֶם אֶת מֵי הַמַּבּוּל. עֲשָׂרָה דוֹרוֹת מִנֹּחַ וְעַד אַבְרָהָם, לְהוֹדִיעַ כַּמָּה אֶרֶךְ אַפַּיִם לְפָנָיו, שֶׁכָּל הַדּוֹרוֹת הָיוּ מַכְעִיסִין וּבָאִין, עַד שֶׁבָּא אַבְרָהָם וְקִבֵּל עָלָיו שְׂכַר כֻּלָּם:
עֲשָׂרָה נִסְיוֹנוֹת נִתְנַסָּה אַבְרָהָם אָבִינוּ עָלָיו הַשָּׁלוֹם וְעָמַד בְּכֻלָּם, לְהוֹדִיעַ כַּמָּה חִבָּתוֹ שֶׁל אַבְרָהָם אָבִינוּ עָלָיו הַשָּׁלוֹם:
עֲשָׂרָה נִסִּים נַעֲשׂוּ לַאֲבוֹתֵינוּ בְמִצְרַיִם וַעֲשָׂרָה עַל הַיָּם. עֶשֶׂר מַכּוֹת הֵבִיא הַקָּדוֹשׁ בָּרוּךְ הוּא עַל הַמִּצְרִיִּים בְּמִצְרַיִם וְעֶשֶׂר עַל הַיָּם. עֲשָׂרָה נִסְיוֹנוֹת נִסּוּ אֲבוֹתֵינוּ אֶת הַמָּקוֹם בָּרוּךְ הוּא בַמִּדְבָּר, שֶׁנֶּאֱמַר (במדבר יד) וַיְנַסּוּ אֹתִי זֶה עֶשֶׂר פְּעָמִים וְלֹא שָׁמְעוּ בְּקוֹלִי:
עֲשָׂרָה נִסִּים נַעֲשׂוּ לַאֲבוֹתֵינוּ בְּבֵית הַמִּקְדָּשׁ. לֹא הִפִּילָה אִשָּׁה מֵרֵיחַ בְּשַׂר הַקֹּדֶשׁ, וְלֹא הִסְרִיחַ בְּשַׂר הַקֹּדֶשׁ מֵעוֹלָם, וְלֹא נִרְאָה זְבוּב בְּבֵית הַמִּטְבָּחַיִם, וְלֹא אֵרַע קֶרִי לְכֹהֵן גָּדוֹל בְּיוֹם הַכִּפּוּרִים, וְלֹא כִבּוּ גְשָׁמִים אֵשׁ שֶׁל עֲצֵי הַמַּעֲרָכָה, וְלֹא נָצְחָה הָרוּחַ אֶת עַמּוּד הֶעָשָׁן, וְלֹא נִמְצָא פְסוּל בָּעֹמֶר וּבִשְׁתֵּי הַלֶּחֶם וּבְלֶחֶם הַפָּנִים, עוֹמְדִים צְפוּפִים וּמִשְׁתַּחֲוִים רְוָחִים, וְלֹא הִזִּיק נָחָשׁ וְעַקְרָב בִּירוּשָׁלַיִם מֵעוֹלָם, וְלֹא אָמַר אָדָם לַחֲבֵרוֹ צַר לִי הַמָּקוֹם שֶׁאָלִין בִּירוּשָׁלַיִם:
עֲשָׂרָה דְבָרִים נִבְרְאוּ בְּעֶרֶב שַׁבָּת בֵּין הַשְּׁמָשׁוֹת, וְאֵלּוּ הֵן, פִּי הָאָרֶץ, וּפִי הַבְּאֵר, וּפִי הָאָתוֹן, וְהַקֶּשֶׁת, וְהַמָּן, וְהַמַּטֶּה, וְהַשָּׁמִיר, וְהַכְּתָב, וְהַמִּכְתָּב, וְהַלּוּחוֹת. וְיֵשׁ אוֹמְרִים, אַף הַמַּזִּיקִין, וּקְבוּרָתוֹ שֶׁל משֶׁה, וְאֵילוֹ שֶׁל אַבְרָהָם אָבִינוּ. וְיֵשׁ אוֹמְרִים, אַף צְבָת בִּצְבָת עֲשׂוּיָה:
שִׁבְעָה דְבָרִים בַּגֹּלֶם וְשִׁבְעָה בֶחָכָם. חָכָם אֵינוֹ מְדַבֵּר בִּפְנֵי מִי שֶׁהוּא גָדוֹל מִמֶּנּוּ בְחָכְמָה וּבְמִנְיָן, וְאֵינוֹ נִכְנָס לְתוֹךְ דִּבְרֵי חֲבֵרוֹ, וְאֵינוֹ נִבְהָל לְהָשִׁיב, שׁוֹאֵל כָּעִנְיָן וּמֵשִׁיב כַּהֲלָכָה, וְאוֹמֵר עַל רִאשׁוֹן רִאשׁוֹן וְעַל אַחֲרוֹן אַחֲרוֹן, וְעַל מַה שֶּׁלֹּא שָׁמַע, אוֹמֵר לֹא שָׁמָעְתִּי, וּמוֹדֶה עַל הָאֱמֶת. וְחִלּוּפֵיהֶן בַּגֹּלֶם:
שִׁבְעָה מִינֵי פֻרְעָנֻיּוֹת בָּאִין לָעוֹלָם עַל שִׁבְעָה גוּפֵי עֲבֵרָה. מִקְצָתָן מְעַשְּׂרִין וּמִקְצָתָן אֵינָן מְעַשְּׂרִין, רָעָב שֶׁל בַּצֹּרֶת בָּאָה, מִקְצָתָן רְעֵבִים וּמִקְצָתָן שְׂבֵעִים. גָּמְרוּ שֶׁלֹּא לְעַשֵּׂר, רָעָב שֶׁל מְהוּמָה וְשֶׁל בַּצֹּרֶת בָּאָה. וְשֶׁלֹּא לִטֹּל אֶת הַחַלָּה, רָעָב שֶׁל כְּלָיָה בָאָה. דֶּבֶר בָּא לָעוֹלָם עַל מִיתוֹת הָאֲמוּרוֹת בַּתּוֹרָה שֶׁלֹּא נִמְסְרוּ לְבֵית דִּין, וְעַל פֵּרוֹת שְׁבִיעִית. חֶרֶב בָּאָה לָעוֹלָם עַל עִנּוּי הַדִּין, וְעַל עִוּוּת הַדִּין, וְעַל הַמּוֹרִים בַּתּוֹרָה שֶׁלֹּא כַהֲלָכָה:
חַיָּה רָעָה בָאָה לָעוֹלָם עַל שְׁבוּעַת שָׁוְא, וְעַל חִלּוּל הַשֵּׁם. גָּלוּת בָּאָה לָעוֹלָם עַל עוֹבְדֵי עֲבוֹדָה זָרָה, וְעַל גִלּוּי עֲרָיוֹת, וְעַל שְׁפִיכוּת דָּמִים, וְעַל הַשְׁמָטַת הָאָרֶץ. בְּאַרְבָּעָה פְרָקִים הַדֶּבֶר מִתְרַבֶּה, בָּרְבִיעִית, וּבַשְּׁבִיעִית, וּבְמוֹצָאֵי שְׁבִיעִית, וּבְמוֹצָאֵי הֶחָג שֶׁבְּכָל שָׁנָה וְשָׁנָה. בָּרְבִיעִית, מִפְּנֵי מַעְשַׂר עָנִי שֶׁבַּשְּׁלִישִׁית. בַּשְּׁבִיעִית, מִפְּנֵי מַעְשַׂר עָנִי שֶׁבַּשִּׁשִּׁית. וּבְמוֹצָאֵי שְׁבִיעִית, מִפְּנֵי פֵרוֹת שְׁבִיעִית. וּבְמוֹצָאֵי הֶחָג שֶׁבְּכָל שָׁנָה וְשָׁנָה, מִפְּנֵי גֶזֶל מַתְּנוֹת עֲנִיִּים:
אַרְבַּע מִדּוֹת בָּאָדָם. הָאוֹמֵר שֶׁלִּי שֶׁלִּי וְשֶׁלְּךָ שֶׁלָּךְ, זוֹ מִדָּה בֵינוֹנִית. וְיֵשׁ אוֹמְרִים, זוֹ מִדַּת סְדוֹם. שֶׁלִּי שֶׁלְּךָ וְשֶׁלְּךָ שֶׁלִּי, עַם הָאָרֶץ. שֶׁלִּי שֶׁלְּךָ וְשֶׁלְּךָ שֶׁלָּךְ, חָסִיד. שֶׁלִּי שֶׁלִּי וְשֶׁלְּךָ שֶׁלִּי, רָשָׁע:
אַרְבַּע מִדּוֹת בַּדֵּעוֹת. נוֹחַ לִכְעֹס וְנוֹחַ לִרְצוֹת, יָצָא שְׂכָרוֹ בְהֶפְסֵדוֹ. קָשֶׁה לִכְעֹס וְקָשֶׁה לִרְצוֹת, יָצָא הֶפְסֵדוֹ בִשְׂכָרוֹ. קָשֶׁה לִכְעֹס וְנוֹחַ לִרְצוֹת, חָסִיד. נוֹחַ לִכְעֹס וְקָשֶׁה לִרְצוֹת, רָשָׁע:
אַרְבַּע מִדּוֹת בַּתַּלְמִידִים. מַהֵר לִשְׁמֹעַ וּמַהֵר לְאַבֵּד, יָצָא שְׂכָרוֹ בְהֶפְסֵדוֹ. קָשֶׁה לִשְׁמֹעַ וְקָשֶׁה לְאַבֵּד, יָצָא הֶפְסֵדוֹ בִשְׂכָרוֹ. מַהֵר לִשְׁמֹעַ וְקָשֶׁה לְאַבֵּד, חָכָם. קָשֶׁה לִשְׁמֹעַ וּמַהֵר לְאַבֵּד, זֶה חֵלֶק רָע:
אַרְבַּע מִדּוֹת בְּנוֹתְנֵי צְדָקָה. הָרוֹצֶה שֶׁיִּתֵּן וְלֹא יִתְּנוּ אֲחֵרִים, עֵינוֹ רָעָה בְּשֶׁל אֲחֵרִים. יִתְּנוּ אֲחֵרִים וְהוּא לֹא יִתֵּן, עֵינוֹ רָעָה בְשֶׁלּוֹ. יִתֵּן וְיִתְּנוּ אֲחֵרִים, חָסִיד. לֹא יִתֵּן וְלֹא יִתְּנוּ אֲחֵרִים, רָשָׁע:
אַרְבַּע מִדּוֹת בְּהוֹלְכֵי לְבֵית הַמִּדְרָשׁ. הוֹלֵךְ וְאֵינוֹ עוֹשֶׂה, שְׂכַר הֲלִיכָה בְיָדוֹ. עוֹשֶׂה וְאֵינוֹ הוֹלֵךְ, שְׂכַר מַעֲשֶׂה בְיָדוֹ. הוֹלֵךְ וְעוֹשֶׂה, חָסִיד. לֹא הוֹלֵךְ וְלֹא עוֹשֶׂה, רָשָׁע:
אַרְבַּע מִדּוֹת בְּיוֹשְׁבִים לִפְנֵי חֲכָמִים. סְפוֹג, וּמַשְׁפֵּךְ, מְשַׁמֶּרֶת, וְנָפָה. סְפוֹג, שֶׁהוּא סוֹפֵג אֶת הַכֹּל. מַשְׁפֵּךְ, שֶׁמַּכְנִיס בְּזוֹ וּמוֹצִיא בְזוֹ. מְשַׁמֶּרֶת, שֶׁמּוֹצִיאָה אֶת הַיַּיִן וְקוֹלֶטֶת אֶת הַשְּׁמָרִים. וְנָפָה, שֶׁמּוֹצִיאָה אֶת הַקֶּמַח וְקוֹלֶטֶת אֶת הַסֹּלֶת:
כָּל אַהֲבָה שֶׁהִיא תְלוּיָה בְדָבָר, בָּטֵל דָּבָר, בְּטֵלָה אַהֲבָה. וְשֶׁאֵינָהּ תְּלוּיָה בְדָבָר, אֵינָהּ בְּטֵלָה לְעוֹלָם. אֵיזוֹ הִיא אַהֲבָה הַתְּלוּיָה בְדָבָר, זוֹ אַהֲבַת אַמְנוֹן וְתָמָר. וְשֶׁאֵינָהּ תְּלוּיָה בְדָבָר, זוֹ אַהֲבַת דָּוִד וִיהוֹנָתָן:
כָּל מַחֲלֹקֶת שֶׁהִיא לְשֵׁם שָׁמַיִם, סוֹפָהּ לְהִתְקַיֵּם. וְשֶׁאֵינָהּ לְשֵׁם שָׁמַיִם, אֵין סוֹפָהּ לְהִתְקַיֵּם. אֵיזוֹ הִיא מַחֲלֹקֶת שֶׁהִיא לְשֵׁם שָׁמַיִם, זוֹ מַחֲלֹקֶת הִלֵּל וְשַׁמַּאי. וְשֶׁאֵינָהּ לְשֵׁם שָׁמַיִם, זוֹ מַחֲלֹקֶת קֹרַח וְכָל עֲדָתוֹ:
כָּל הַמְזַכֶּה אֶת הָרַבִּים, אֵין חֵטְא בָּא עַל יָדוֹ. וְכָל הַמַּחֲטִיא אֶת הָרַבִּים, אֵין מַסְפִּיקִין בְּיָדוֹ לַעֲשׂוֹת תְּשׁוּבָה. משֶׁה זָכָה וְזִכָּה אֶת הָרַבִּים, זְכוּת הָרַבִּים תָּלוּי בּוֹ, שֶׁנֶּאֱמַר (דברים לג) צִדְקַת ה' עָשָׂה וּמִשְׁפָּטָיו עִם יִשְׂרָאֵל. יָרָבְעָם חָטָא וְהֶחֱטִיא אֶת הָרַבִּים, חֵטְא הָרַבִּים תָּלוּי בּוֹ, שֶׁנֶּאֱמַר (מלכים א טו) עַל חַטֹּאות יָרָבְעָם (בֶּן נְבָט) אֲשֶׁר חָטָא וַאֲשֶׁר הֶחֱטִיא אֶת יִשְׂרָאֵל:
כָּל מִי שֶׁיֵּשׁ בְּיָדוֹ שְׁלשָׁה דְבָרִים הַלָּלוּ, מִתַּלְמִידָיו שֶׁל אַבְרָהָם אָבִינוּ. וּשְׁלשָׁה דְבָרִים אֲחֵרִים, מִתַּלְמִידָיו שֶׁל בִּלְעָם הָרָשָׁע. עַיִן טוֹבָה, וְרוּחַ נְמוּכָה, וְנֶפֶשׁ שְׁפָלָה, מִתַּלְמִידָיו שֶׁל אַבְרָהָם אָבִינוּ. עַיִן רָעָה, וְרוּחַ גְּבוֹהָה, וְנֶפֶשׁ רְחָבָה, מִתַּלְמִידָיו שֶׁל בִּלְעָם הָרָשָׁע. מַה בֵּין תַּלְמִידָיו שֶׁל אַבְרָהָם אָבִינוּ לְתַלְמִידָיו שֶׁל בִּלְעָם הָרָשָׁע. תַּלְמִידָיו שֶׁל אַבְרָהָם אָבִינוּ, אוֹכְלִין בָּעוֹלָם הַזֶּה וְנוֹחֲלִין בָּעוֹלָם הַבָּא, שֶׁנֶּאֱמַר (משלי ח) לְהַנְחִיל אֹהֲבַי יֵשׁ, וְאֹצְרֹתֵיהֶם אֲמַלֵּא. אֲבָל תַּלְמִידָיו שֶׁל בִּלְעָם הָרָשָׁע יוֹרְשִׁין גֵּיהִנֹּם וְיוֹרְדִין לִבְאֵר שַׁחַת, שֶׁנֶּאֱמַר (תהלים נה) וְאַתָּה אֱלֹהִים תּוֹרִידֵם לִבְאֵר שַׁחַת, אַנְשֵׁי דָמִים וּמִרְמָה לֹא יֶחֱצוּ יְמֵיהֶם, וַאֲנִי אֶבְטַח בָּךְ:
יְהוּדָה בֶן תֵּימָא אוֹמֵר, הֱוֵי עַז כַּנָּמֵר, וְקַל כַּנֶּשֶׁר, וְרָץ כַּצְּבִי, וְגִבּוֹר כָּאֲרִי, לַעֲשׂוֹת רְצוֹן אָבִיךָ שֶׁבַּשָּׁמָיִם. הוּא הָיָה אוֹמֵר, עַז פָּנִים לְגֵיהִנֹּם, וּבֹשֶׁת פָּנִים לְגַן עֵדֶן. יְהִי רָצוֹן מִלְּפָנֶיךָ יְיָ אֱלֹהֵינוּ שֶׁתִּבְנֶה עִירְךָ בִּמְהֵרָה בְיָמֵינוּ וְתֵן חֶלְקֵנוּ בְתוֹרָתֶךָ:
הוּא הָיָה אוֹמֵר, בֶּן חָמֵשׁ שָׁנִים לַמִּקְרָא, בֶּן עֶשֶׂר לַמִּשְׁנָה, בֶּן שְׁלשׁ עֶשְׂרֵה לַמִּצְוֹת, בֶּן חֲמֵשׁ עֶשְׂרֵה לַתַּלְמוּד, בֶּן שְׁמֹנֶה עֶשְׂרֵה לַחֻפָּה, בֶּן עֶשְׂרִים לִרְדֹּף, בֶּן שְׁלשִׁים לַכֹּחַ, בֶּן אַרְבָּעִים לַבִּינָה, בֶּן חֲמִשִּׁים לָעֵצָה, בֶּן שִׁשִּׁים לַזִּקְנָה, בֶּן שִׁבְעִים לַשֵּׂיבָה, בֶּן שְׁמֹנִים לַגְּבוּרָה, בֶּן תִּשְׁעִים לָשׁוּחַ, בֶּן מֵאָה כְּאִלּוּ מֵת וְעָבַר וּבָטֵל מִן הָעוֹלָם:
בֶּן בַּג בַּג אוֹמֵר, הֲפֹךְ בָּהּ וַהֲפֹךְ בָּהּ, דְּכֹלָּא בָהּ. וּבָהּ תֶּחֱזֵי, וְסִיב וּבְלֵה בָהּ, וּמִנַּהּ לֹא תָזוּעַ, שֶׁאֵין לְךָ מִדָּה טוֹבָה הֵימֶנָּה:
בֶּן הֵא הֵא אוֹמֵר, לְפוּם צַעֲרָא אַגְרָא:

Chapter 6

שָׁנוּ חֲכָמִים בִּלְשׁוֹן הַמִּשְׁנָה, בָּרוּךְ שֶׁבָּחַר בָּהֶם וּבְמִשְׁנָתָם: רַבִּי מֵאִיר אוֹמֵר כָּל הָעוֹסֵק בַּתּוֹרָה לִשְׁמָהּ, זוֹכֶה לִדְבָרִים הַרְבֵּה. וְלֹא עוֹד אֶלָּא שֶׁכָּל הָעוֹלָם כֻּלּוֹ כְדַי הוּא לוֹ. נִקְרָא רֵעַ, אָהוּב, אוֹהֵב אֶת הַמָּקוֹם, אוֹהֵב אֶת הַבְּרִיּוֹת, מְשַׂמֵּחַ אֶת הַמָּקוֹם, מְשַׂמֵּחַ אֶת הַבְּרִיּוֹת. וּמַלְבַּשְׁתּוֹ עֲנָוָה וְיִרְאָה, וּמַכְשַׁרְתּוֹ לִהְיוֹת צַדִּיק וְחָסִיד וְיָשָׁר וְנֶאֱמָן, וּמְרַחַקְתּוֹ מִן הַחֵטְא, וּמְקָרַבְתּוֹ לִידֵי זְכוּת, וְנֶהֱנִין מִמֶּנּוּ עֵצָה וְתוּשִׁיָּה בִּינָה וּגְבוּרָה, שֶׁנֶּאֱמַר (משלי ח) לִי עֵצָה וְתוּשִׁיָּה אֲנִי בִינָה לִי גְבוּרָה. וְנוֹתֶנֶת לוֹ מַלְכוּת וּמֶמְשָׁלָה וְחִקּוּר דִּין, וּמְגַלִּין לוֹ רָזֵי תוֹרָה, וְנַעֲשֶׂה כְמַעְיָן הַמִּתְגַּבֵּר וּכְנָהָר שֶׁאֵינוֹ פוֹסֵק, וֶהֱוֵי צָנוּעַ וְאֶרֶךְ רוּחַ, וּמוֹחֵל עַל עֶלְבּוֹנוֹ, וּמְגַדַּלְתּוֹ וּמְרוֹמַמְתּוֹ עַל כָּל הַמַּעֲשִׂים:
אָמַר רַבִּי יְהוֹשֻׁעַ בֶּן לֵוִי, בְּכָל יוֹם וָיוֹם בַּת קוֹל יוֹצֵאת מֵהַר חוֹרֵב וּמַכְרֶזֶת וְאוֹמֶרֶת, אוֹי לָהֶם לַבְּרִיּוֹת מֵעֶלְבּוֹנָהּ שֶׁל תּוֹרָה. שֶׁכָּל מִי שֶׁאֵינוֹ עוֹסֵק בַּתּוֹרָה נִקְרָא נָזוּף, שֶׁנֶּאֱמַר (משלי יא) נֶזֶם זָהָב בְּאַף חֲזִיר אִשָּׁה יָפָה וְסָרַת טָעַם. וְאוֹמֵר (שמות לב) וְהַלֻּחֹת מַעֲשֵׂה אֱלֹהִים הֵמָּה וְהַמִּכְתָּב מִכְתַּב אֱלֹהִים הוּא חָרוּת עַל הַלֻּחֹת, אַל תִּקְרָא חָרוּת אֶלָּא חֵרוּת, שֶׁאֵין לְךָ בֶן חוֹרִין אֶלָּא מִי שֶׁעוֹסֵק בְּתַלְמוּד תּוֹרָה. וְכָל מִי שֶׁעוֹסֵק בְּתַלְמוּד תּוֹרָה הֲרֵי זֶה מִתְעַלֶּה, שֶׁנֶּאֱמַר (במדבר כא) וּמִמַּתָּנָה נַחֲלִיאֵל וּמִנַּחֲלִיאֵל בָּמוֹת:
הַלּוֹמֵד מֵחֲבֵרוֹ פֶּרֶק אֶחָד אוֹ הֲלָכָה אַחַת אוֹ פָסוּק אֶחָד אוֹ דִבּוּר אֶחָד אוֹ אֲפִלּוּ אוֹת אַחַת, צָרִיךְ לִנְהוֹג בּוֹ כָבוֹד, שֶׁכֵּן מָצִינוּ בְדָוִד מֶלֶךְ יִשְׂרָאֵל, שֶׁלֹּא לָמַד מֵאֲחִיתֹפֶל אֶלָּא שְׁנֵי דְבָרִים בִּלְבָד, קְרָאוֹ רַבּוֹ אַלּוּפוֹ וּמְיֻדָּעוֹ, שֶׁנֶּאֱמַר (תהלים נה) וְאַתָּה אֱנוֹשׁ כְּעֶרְכִּי אַלּוּפִי וּמְיֻדָּעִי. וַהֲלֹא דְבָרִים קַל וָחֹמֶר, וּמַה דָּוִד מֶלֶךְ יִשְׂרָאֵל, שֶׁלֹּא לָמַד מֵאֲחִיתֹפֶל אֶלָּא שְׁנֵי דְבָרִים בִּלְבַד קְרָאוֹ רַבּוֹ אַלּוּפוֹ וּמְיֻדָּעוֹ, הַלּוֹמֵד מֵחֲבֵרוֹ פֶּרֶק אֶחָד אוֹ הֲלָכָה אַחַת אוֹ פָסוּק אֶחָד אוֹ דִבּוּר אֶחָד אוֹ אֲפִלּוּ אוֹת אַחַת, עַל אַחַת כַּמָּה וְכַמָּה שֶׁצָּרִיךְ לִנְהוֹג בּוֹ כָבוֹד. וְאֵין כָּבוֹד אֶלָּא תוֹרָה, שֶׁנֶּאֱמַר (משלי ג) כָּבוֹד חֲכָמִים יִנְחָלוּ, (משלי כח) וּתְמִימִים יִנְחֲלוּ טוֹב, וְאֵין טוֹב אֶלָּא תוֹרָה, שֶׁנֶּאֱמַר (משלי ד) כִּי לֶקַח טוֹב נָתַתִּי לָכֶם תּוֹרָתִי אַל תַּעֲזֹבוּ:
כַּךְ הִיא דַּרְכָּהּ שֶׁל תּוֹרָה, פַּת בְּמֶלַח תֹּאכַל, וּמַיִם בִּמְשׂוּרָה תִשְׁתֶּה, וְעַל הָאָרֶץ תִּישַׁן, וְחַיֵּי צַעַר תִּחְיֶה, וּבַתּוֹרָה אַתָּה עָמֵל, אִם אַתָּה עֹשֶׂה כֵן, (תהלים קכח) אַשְׁרֶיךָ וְטוֹב לָךְ. אַשְׁרֶיךָ בָּעוֹלָם הַזֶּה וְטוֹב לָךְ לָעוֹלָם הַבָּא:
אַל תְּבַקֵּשׁ גְּדֻלָּה לְעַצְמְךָ, וְאַל תַּחְמֹד כָּבוֹד, יוֹתֵר מִלִּמּוּדְךָ עֲשֵׂה, וְאַל תִּתְאַוֶּה לְשֻׁלְחָנָם שֶׁל מְלָכִים, שֶׁשֻּׁלְחָנְךָ גָדוֹל מִשֻּׁלְחָנָם, וְכִתְרְךָ גָדוֹל מִכִּתְרָם, וְנֶאֱמָן הוּא בַּעַל מְלַאכְתְּךָ שֶׁיְּשַׁלֵּם לְךָ שְׂכַר פְּעֻלָּתֶךָ:
גְּדוֹלָה תוֹרָה יוֹתֵר מִן הַכְּהֻנָּה וּמִן הַמַּלְכוּת, שֶׁהַמַּלְכוּת נִקְנֵית בִּשְׁלֹשִׁים מַעֲלוֹת, וְהַכְּהֻנָּה בְּעֶשְׂרִים וְאַרְבַּע, וְהַתּוֹרָה נִקְנֵית בְּאַרְבָּעִים וּשְׁמֹנָה דְבָרִים. וְאֵלוּ הֵן, בְּתַלְמוּד, בִּשְׁמִיעַת הָאֹזֶן, בַּעֲרִיכַת שְׂפָתַיִם, בְּבִינַת הַלֵּב, בְּשִׂכְלוּת הַלֵּב, בְּאֵימָה, בְּיִרְאָה, בַּעֲנָוָה, בְּשִׂמְחָה, בְּטָהֳרָה, בְּשִׁמּוּשׁ חֲכָמִים, בְּדִקְדּוּק חֲבֵרִים, וּבְפִלְפּוּל הַתַּלְמִידִים, בְּיִשּׁוּב, בַּמִּקְרָא, בַּמִּשְׁנָה, בְּמִעוּט סְחוֹרָה, בְּמִעוּט דֶּרֶךְ אֶרֶץ, בְּמִעוּט תַּעֲנוּג, בְּמִעוּט שֵׁינָה, בְּמִעוּט שִׂיחָה, בְּמִעוּט שְׂחוֹק, בְּאֶרֶךְ אַפַּיִם, בְּלֵב טוֹב, בֶּאֱמוּנַת חֲכָמִים, וּבְקַבָּלַת הַיִּסּוּרִין, הַמַּכִּיר אֶת מְקוֹמוֹ, וְהַשָּׂמֵחַ בְּחֶלְקוֹ, וְהָעוֹשֶׂה סְיָג לִדְבָרָיו, וְאֵינוֹ מַחֲזִיק טוֹבָה לְעַצְמוֹ, אָהוּב, אוֹהֵב אֶת הַמָּקוֹם, אוֹהֵב אֶת הַבְּרִיּוֹת, אוֹהֵב אֶת הַצְּדָקוֹת, אוֹהֵב אֶת הַמֵּישָׁרִים, אוֹהֵב אֶת הַתּוֹכָחוֹת, מִתְרַחֵק מִן הַכָּבוֹד, וְלֹא מֵגִיס לִבּוֹ בְתַלְמוּדוֹ, וְאֵינוֹ שָׂמֵחַ בְּהוֹרָאָה, נוֹשֵׂא בְעֹל עִם חֲבֵרוֹ, מַכְרִיעוֹ לְכַף זְכוּת, מַעֲמִידוֹ עַל הָאֱמֶת, וּמַעֲמִידוֹ עַל הַשָּׁלוֹם, מִתְיַשֵּׁב לִבּוֹ בְתַלְמוּדוֹ, שׁוֹאֵל וּמֵשִׁיב, שׁוֹמֵעַ וּמוֹסִיף, הַלּוֹמֵד עַל מְנָת לְלַמֵּד וְהַלּוֹמֵד עַל מְנָת לַעֲשׂוֹת, הַמַּחְכִּים אֶת רַבּוֹ, וְהַמְכַוֵּן אֶת שְׁמוּעָתוֹ, וְהָאוֹמֵר דָּבָר בְּשֵׁם אוֹמְרוֹ, הָא לָמַדְתָּ שֶׁכָּל הָאוֹמֵר דָּבָר בְּשֵׁם אוֹמְרוֹ מֵבִיא גְאֻלָּה לָעוֹלָם, שֶׁנֶּאֱמַר (אסתר ב) וַתֹּאמֶר אֶסְתֵּר לַמֶּלֶךְ בְּשֵׁם מָרְדֳּכָי:
גְּדוֹלָה תוֹרָה שֶׁהִיא נוֹתֶנֶת חַיִּים לְעֹשֶׂיהָ בָּעוֹלָם הַזֶּה וּבָעוֹלָם הַבָּא, שֶׁנֶּאֱמַר (משלי ד) כִּי חַיִּים הֵם לְמֹצְאֵיהֶם וּלְכָל בְּשָׂרוֹ מַרְפֵּא. וְאוֹמֵר (שם ג) רִפְאוּת תְּהִי לְשָׁרֶךָ וְשִׁקּוּי לְעַצְמוֹתֶיךָ. וְאוֹמֵר (שם ג) עֵץ חַיִּים הִיא לַמַּחֲזִיקִים בָּהּ וְתֹמְכֶיהָ מְאֻשָּׁר. וְאוֹמֵר (שם א) כִּי לִוְיַת חֵן הֵם לְרֹאשֶׁךָ וַעֲנָקִים לְגַרְגְּרֹתֶיךָ. וְאוֹמֵר (שם ד) תִּתֵּן לְרֹאשְׁךָ לִוְיַת חֵן עֲטֶרֶת תִּפְאֶרֶת תְּמַגְּנֶךָּ. וְאוֹמֵר (שם ט) כִּי בִי יִרְבּוּ יָמֶיךָ וְיוֹסִיפוּ לְךָ שְׁנוֹת חַיִּים. וְאוֹמֵר (שם ג) אֹרֶךְ יָמִים בִּימִינָהּ בִּשְׂמֹאולָהּ עֹשֶׁר וְכָבוֹד. וְאוֹמֵר (שם) כִּי אֹרֶךְ יָמִים וּשְׁנוֹת חַיִּים וְשָׁלוֹם יוֹסִיפוּ לָךְ. וְאוֹמֵר (שם) דְּרָכֶיהָ דַּרְכֵי נֹעַם וְכָל נְתִיבוֹתֶיהָ שָׁלוֹם:
רַבִּי שִׁמְעוֹן בֶּן יְהוּדָה מִשּׁוּם רַבִּי שִׁמְעוֹן בֶּן יוֹחַאי אוֹמֵר, הַנּוֹי וְהַכֹּחַ וְהָעֹשֶׁר וְהַכָּבוֹד וְהַחָכְמָה וְהַזִּקְנָה וְהַשֵּׂיבָה וְהַבָּנִים, נָאֶה לַצַּדִּיקִים וְנָאֶה לָעוֹלָם, שֶׁנֶּאֱמַר (שם טז) עֲטֶרֶת תִּפְאֶרֶת שֵׂיבָה בְּדֶרֶךְ צְדָקָה תִּמָּצֵא. וְאוֹמֵר (שם כ) תִּפְאֶרֶת בַּחוּרִים כֹּחָם וַהֲדַר זְקֵנִים שֵׂיבָה. וְאוֹמֵר (שם יד) עֲטֶרֶת חֲכָמִים עָשְׁרָם. וְאוֹמֵר (שם יז) עֲטֶרֶת זְקֵנִים בְּנֵי בָנִים וְתִפְאֶרֶת בָּנִים אֲבוֹתָם. וְאוֹמֵר (ישעיה כד) וְחָפְרָה הַלְּבָנָה וּבוֹשָׁה הַחַמָּה, כִּי מָלַךְ ה' צְבָאוֹת בְּהַר צִיּוֹן וּבִירוּשָׁלַיִם וְנֶגֶד זְקֵנָיו כָּבוֹד. רַבִּי שִׁמְעוֹן בֶּן מְנַסְיָא אוֹמֵר, אֵלּוּ שֶׁבַע מִדּוֹת שֶׁמָּנוּ חֲכָמִים לַצַּדִּיקִים, כֻּלָּם נִתְקַיְּמוּ בְרַבִּי וּבְבָנָיו:
אָמַר רַבִּי יוֹסֵי בֶן קִסְמָא, פַּעַם אַחַת הָיִיתִי מְהַלֵּךְ בַּדֶּרֶךְ וּפָגַע בִּי אָדָם אֶחָד, וְנָתַן לִי שָׁלוֹם, וְהֶחֱזַרְתִּי לוֹ שָׁלוֹם. אָמַר לִי, רַבִּי, מֵאֵיזֶה מָקוֹם אַתָּה. אָמַרְתִּי לוֹ, מֵעִיר גְּדוֹלָה שֶׁל חֲכָמִים וְשֶׁל סוֹפְרִים אָנִי. אָמַר לִי, רַבִּי, רְצוֹנְךָ שֶׁתָּדוּר עִמָּנוּ בִמְקוֹמֵנוּ, וַאֲנִי אֶתֵּן לְךָ אֶלֶף אֲלָפִים דִּינְרֵי זָהָב וַאֲבָנִים טוֹבוֹת וּמַרְגָּלִיּוֹת. אָמַרְתִּי לוֹ, בְּנִי, אִם אַתָּה נוֹתֵן לִי כָל כֶּסֶף וְזָהָב וַאֲבָנִים טוֹבוֹת וּמַרְגָּלִיּוֹת שֶׁבָּעוֹלָם, אֵינִי דָר אֶלָּא בִמְקוֹם תּוֹרָה. וְלֹא עוֹד, אֶלָּא שֶׁבִּשְׁעַת פְּטִירָתוֹ שֶׁל אָדָם אֵין מְלַוִּין לוֹ לָאָדָם לֹא כֶסֶף וְלֹא זָהָב וְלֹא אֲבָנִים טוֹבוֹת וּמַרְגָּלִיּוֹת, אֶלָּא תוֹרָה וּמַעֲשִׂים טוֹבִים בִּלְבַד, שֶׁנֶּאֱמַר (משלי ו) בְּהִתְהַלֶּכְךָ תַּנְחֶה אֹתָךְ, בְּשָׁכְבְּךָ תִּשְׁמֹר עָלֶיךָ, וַהֲקִיצוֹתָ הִיא תְשִׂיחֶךָ. בְּהִתְהַלֶּכְךָ תַּנְחֶה אֹתָךְ, בָּעוֹלָם הַזֶּה, בְּשָׁכְבְּךָ תִּשְׁמֹר עָלֶיךָ, בַּקֶּבֶר, וַהֲקִיצוֹתָ הִיא תְשִׂיחֶךָ, לָעוֹלָם הַבָּא. וְכֵן כָּתוּב בְּסֵפֶר תְּהִלִּים עַל יְדֵי דָוִד מֶלֶךְ יִשְׂרָאֵל (תהלים קיט), טוֹב לִי תוֹרַת פִּיךָ מֵאַלְפֵי זָהָב וָכָסֶף. וְאוֹמֵר (חגי ב) לִי הַכֶּסֶף וְלִי הַזָּהָב אָמַר ה' צְבָאוֹת:
חֲמִשָּׁה קִנְיָנִים קָנָה לוֹ הַקָּדוֹשׁ בָּרוּךְ הוּא בְעוֹלָמוֹ, וְאֵלּוּ הֵן, תּוֹרָה קִנְיָן אֶחָד, שָׁמַיִם וָאָרֶץ קִנְיָן אֶחָד, אַבְרָהָם קִנְיָן אֶחָד, יִשְׂרָאֵל קִנְיָן אֶחָד, בֵּית הַמִּקְדָּשׁ קִנְיָן אֶחָד. תּוֹרָה מִנַּיִן, דִּכְתִיב (משלי ח), ה' קָנָנִי רֵאשִׁית דַּרְכּוֹ קֶדֶם מִפְעָלָיו מֵאָז. שָׁמַיִם וָאָרֶץ קִנְיָן אֶחָד מִנַּיִן, דִּכְתִיב (ישעיה סו), כֹּה אָמַר ה' הַשָּׁמַיִם כִּסְאִי וְהָאָרֶץ הֲדֹם רַגְלָי אֵי זֶה בַיִת אֲשֶׁר תִּבְנוּ לִי וְאֵי זֶה מָקוֹם מְנוּחָתִי, וְאוֹמֵר (תהלים קד) מָה רַבּוּ מַעֲשֶׂיךָ ה' כֻּלָּם בְּחָכְמָה עָשִׂיתָ מָלְאָה הָאָרֶץ קִנְיָנֶךָ. אַבְרָהָם קִנְיָן אֶחָד מִנַּיִן, דִּכְתִיב (בראשית יד), וַיְבָרְכֵהוּ וַיֹּאמַר בָּרוּךְ אַבְרָם לְאֵל עֶלְיוֹן קֹנֵה שָׁמַיִם וָאָרֶץ. יִשְׂרָאֵל קִנְיָן אֶחָד מִנַּיִן, דִּכְתִיב (שמות טו), עַד יַעֲבֹר עַמְּךָ ה' עַד יַעֲבֹר עַם זוּ קָנִיתָ, וְאוֹמֵר (תהלים טז) לִקְדוֹשִׁים אֲשֶׁר בָּאָרֶץ הֵמָּה וְאַדִּירֵי כָּל חֶפְצִי בָם. בֵּית הַמִּקְדָּשׁ קִנְיָן אֶחָד מִנַּיִן, דִּכְתִיב (שמות טו), מָכוֹן לְשִׁבְתְּךָ פָּעַלְתָּ ה' מִקְּדָשׁ ה' כּוֹנְנוּ יָדֶיךָ. וְאוֹמֵר (תהלים עח) וַיְבִיאֵם אֶל גְּבוּל קָדְשׁוֹ הַר זֶה קָנְתָה יְמִינוֹ:
כָּל מַה שֶּׁבָּרָא הַקָּדוֹשׁ בָּרוּךְ הוּא בְּעוֹלָמוֹ, לֹא בְרָאוֹ אֶלָּא לִכְבוֹדוֹ, שֶׁנֶּאֱמַר (ישעיה מג), כֹּל הַנִּקְרָא בִשְׁמִי וְלִכְבוֹדִי בְּרָאתִיו יְצַרְתִּיו אַף עֲשִׂיתִיו, וְאוֹמֵר (שמות טו), יְהֹוָה יִמְלֹךְ לְעֹלָם וָעֶד. רַבִּי חֲנַנְיָא בֶּן עֲקַשְׁיָא אוֹמֵר, רָצָה הַקָּדוֹשׁ בָּרוּךְ הוּא לְזַכּוֹת אֶת יִשְׂרָאֵל, לְפִיכָךְ הִרְבָּה לָהֶם תּוֹרָה וּמִצְוֹת, שֶׁנֶּאֱמַר (ישעיה מב) ה' חָפֵץ לְמַעַן צִדְקוֹ יַגְדִּיל תּוֹרָה וְיַאְדִּיר: